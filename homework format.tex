\documentclass{article}%
\usepackage{amsmath}%
\usepackage{amsfonts}%
\usepackage{amssymb}%
\usepackage{graphicx}
\usepackage{listings}
\usepackage{color}\usepackage{listings}
\usepackage{color}
%-------------------------------------------
\newenvironment{proof}[1][Proof]{\textbf{#1.} }{\ \rule{0.5em}{0.5em}}
\setlength{\textwidth}{7.0in}
\setlength{\oddsidemargin}{-0.35in}
\setlength{\topmargin}{-0.5in}
\setlength{\textheight}{9.0in}
\setlength{\parindent}{0.3in}

\definecolor{gray}{rgb}{0.5,0.5,0.5}
\definecolor{mauve}{rgb}{0.58,0,0.82}
\lstset{frame=tb,
  language=C++,
  aboveskip=3mm,
  belowskip=3mm,
  showstringspaces=false,
  columns=flexible,
  basicstyle={\small\ttfamily},
  numbers=left,
  keywordstyle=,
  breaklines=true,m
  breakatwhitespace=true,
  tabsize=3
}

\begin{document}

\begin{flushright}
\textbf{Kai Erik Niermann \\
September 1, 2021}
\end{flushright}

\begin{center}
\textbf{CS - CTfC \\
Homework 1.1: Fibonacci numbers} \\
\end{center}

This solution is limited because it calculates calls its already made. Though in terms of intuitive understanding I think the recursive approach
is better because it directly implements the formula \(F_{n} =F_{n-1} +F_{n-2}\)

\section*{solution}

\begin{lstlisting}[]
  //calculates any Nth fib number
  function fib that takes int N 
        if N <= 1 
        return N
        else 
        return with call fib(N - 1) + fib(N - 2) 

  get input A

  //prints all fib num from 0 to A
  for X times from 0 to A
        print output from call fib with X as parameter
\end{lstlisting}

  
\end{document}